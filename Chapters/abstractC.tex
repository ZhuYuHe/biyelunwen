\begin{abstract}
分拣作业是制造业中非常常见的一种作业场景,由于其重复性强、作业场景单一的任务特性,分拣作业成为工业机器人的重要应用场景之一。传统的自动分拣
系统使用传统图像方法,针对特定的检测工件,人工构造特征,使用模板匹配工件位置,再进行分类。该方法准确率不高,鲁棒性差,可移植性不强。同时,国内
制造业从业者中缺乏图像方面的人才,提高了基于视觉的自动分拣系统在制造业中的应用门槛。因此,针对基于视觉的自动分拣系统的算法效果及其应用门槛问题,
本文设计并实现了基于深度学习的自动分拣系统及用于自动分拣系统目标检测算法训练的云平台。

首先,基于深度学习计算量大的情况及自动分拣系统的具体场景,设计出基于深度学习的自动分拣系统的整体架构,将自动分拣系统的表现层面(抓取工件与获取图像)、控制层面(图像处理和机械臂控制)以及
后台层面(目标检测算法训练与选择)有机地分离开来,使得自动分拣系统的各模块呈现出松耦合的关系,方便后续深度学习云平台的开发;同时,结合自动分拣系统使用基于深度学习
的目标检测算法的情况,针对性地进行了硬件选型、模型选型和通信方式选择。针对摄像头和机械臂,进行了手眼标定和机械臂控制指令方案的设计与实现。

其次,设计自动分拣系统图像处理模块,针对深度学习算法耗时较久的问题,将传统图像处理算法和深度学习目标检测算法进行有机结合,
以降低自动分拣系统图像处理模块的使用延时。同时,由于深度学习模型的训练需要大量数据集,
结合制造业工厂中数据量少,且数据标注费时费力的具体情况,使用迁移学习技术提高了少量标注
数据集下训练模型的表现。然后结合自动分拣系统的实际硬件配置情况,选择YOLOv3-tiny作为自动分拣系统的目标检测算法模型配置。

其三,设计并实现了用于训练深度学习目标检测算法的云平台。结合云平台实际使用场景和用户需求,设计出Web-服务器的开发方案。完成了Web前端页面
和服务器驻留程序的开发。用户通过简单的Web页面的操作,即可获取定制化的深度学习目标检测模型。

最后,针对自动分拣系统的图像处理模块,设计实验验证了目标检测模型配置选择的优越性,收集了自动分拣系统对于工件的抓取成功率和分拣成功率,验证了
自动分拣系统的实用性与稳定性。对深度学习云平台的Web前端页面和服务器后台进行了性能测试,实验结果证明了深度学习云平台具备良好的稳定性。

本文设计的基于深度学习的自动分拣系统,在改进传统自动分拣系统目标检测算法不足的同时,也针对性地进行了硬件选型、模型训练
等方面的优化,同时提出的深度学习云平台大大降低了本系统在制造业中的应用门槛。


\keywords{深度学习,目标检测,分拣系统,云平台,机械臂}
\end{abstract}
